\section{Como funciona?}
	\subsection{Exempo prático}
		\begin{frame}
		\frametitle{\secname}
		\framesubtitle{\subsecname}
				
			\begin{description}
				\item[gap] = tam/FE = $^{10}/_{1.3}$ = 7;
				\item[swap] = false;
				\item[i] = 0; \tikz\node[coordinate] (n2) {};
				\item[j] = i + gap = 7; \tikz\node[coordinate] (n1) {};
			\end{description}

			\begin{table}[!htb]
				\centering
					\begin{tabular}{|l|c|c|c|c|c|c|c|c|c|c|}
						\hline
						\rowcolor{chameleongreen3}Posição     & \tikz\node(t2){0}; & 1 & 2 & 3 & 4 & 5 & 6 & \tikz\node(t1){7}; & 8 & 9\\ \hline
						\rowcolor{chameleongreen2}\cellcolor{chameleongreen3}Valor       & \tikz\node(n3){2}; & 4 & 9 & 0 & 1 & 7 & 5 & \tikz\node(t3){3}; & 8 & 6\\ \hline
					\end{tabular}
			\end{table}
			\begin{tikzpicture}[overlay]
				\path[->] (n1) edge [out=0, in=90] (t1);
				\path[->] (n2) edge [out=0, in=90] (t2);
				\path[->] (n3) edge [out=-90, in=-90] node[below] {\footnotesize{menor}} (t3);
			\end{tikzpicture}
		\end{frame}

		\begin{frame}
		\frametitle{\secname}
		\framesubtitle{\subsecname}
				
			\begin{description}
				\item[gap] = 7;
				\item[swap] = false;
				\item[i] = 1; \tikz\node[coordinate] (n2) {};
				\item[j] = i + gap = 8; \tikz\node[coordinate] (n1) {};
			\end{description}

			\begin{table}[!htb]
				\centering
					\begin{tabular}{|l|c|c|c|c|c|c|c|c|c|c|}
						\hline
						\rowcolor{chameleongreen3}Posição     & 0 & \tikz\node(t2){1}; & 2 & 3 & 4 & 5 & 6 & 7 & \tikz\node(t1){8}; & 9\\ \hline
						\rowcolor{chameleongreen2}\cellcolor{chameleongreen3}Valor       & 2 & \tikz\node(n3){4}; & 9 & 0 & 1 & 7 & 5 & 3 & \tikz\node(t3){8}; & 6\\ \hline
					\end{tabular}
			\end{table}
			\begin{tikzpicture}[overlay]
				\path[->] (n1) edge [out=0, in=90] (t1);
				\path[->] (n2) edge [out=0, in=90] (t2);
				\path[->] (n3) edge [out=-90, in=-90] node[below] {\footnotesize{menor}} (t3);
			\end{tikzpicture}
		\end{frame}
	
		\begin{frame}
		\frametitle{\secname}
		\framesubtitle{\subsecname}
				
			\begin{description}
				\item[gap] = 7;
				\item[swap] = false; 
				\item[i] = 2; \tikz\node[coordinate] (n2) {};
				\item[j] = i + gap = 9; \tikz\node[coordinate] (n1) {};
			\end{description}

			\begin{table}[!htb]
				\centering
					\begin{tabular}{|l|c|c|c|c|c|c|c|c|c|c|}
						\hline
						\rowcolor{chameleongreen3}Posição     & 0 & 1 & \tikz\node(t2){2}; & 3 & 4 & 5 & 6 & 7 & 8 & \tikz\node(t1){9};\\ \hline
						\rowcolor{chameleongreen2}\cellcolor{chameleongreen3}Valor       & 2 & 4 & \tikz\node(n3){9}; & 0 & 1 & 7 & 5 & 3 & 8 & \tikz\node(t3){6};\\ \hline
					\end{tabular}
			\end{table}
			\begin{tikzpicture}[overlay]
				\path[->] (n1) edge [out=0, in=90] (t1);
				\path[->] (n2) edge [out=0, in=90] (t2);
				\path[<->,red] (n3) edge [out=-90, in=-90] node[below] {\footnotesize{troca}} (t3);
			\end{tikzpicture}
		\end{frame}

		\begin{frame}
		\frametitle{\secname}
		\framesubtitle{\subsecname}
				
			\begin{description}
				\item[gap] = 7;
				\item[swap] = true; 
				\item[i] = 2; \tikz\node[coordinate] (n2) {};
				\item[j] = i + gap = 9; \tikz\node[coordinate] (n1) {};
			\end{description}

			\begin{table}[!htb]
				\centering
					\begin{tabular}{|l|c|c|c|c|c|c|c|c|c|c|}
						\hline
						\rowcolor{chameleongreen3}Posição     & 0 & 1 & \tikz\node(t2){2}; & 3 & 4 & 5 & 6 & 7 & 8 & \tikz\node(t1){9};\\ \hline
						\rowcolor{chameleongreen2}\cellcolor{chameleongreen3}Valor       & 2 & 4 & \tikz\node(n3){6}; & 0 & 1 & 7 & 5 & 3 & 8 & \tikz\node(t3){9};\\ \hline
					\end{tabular}
			\end{table}
			\begin{tikzpicture}[overlay]
				\path[->] (n1) edge [out=0, in=90] (t1);
				\path[->] (n2) edge [out=0, in=90] (t2);
				\path[<->,red] (n3) edge [out=-90, in=-90] node[below] {\footnotesize{troca}} (t3);
			\end{tikzpicture}
		\end{frame}

		\begin{frame}
		\frametitle{\secname}
		\framesubtitle{\subsecname}
				
			\begin{description}
				\item[gap] = gap/FE = $^7/_{1.3}$ = 5;
				\item[swap] = false;
				\item[i] = 0; \tikz\node[coordinate] (n2) {};
				\item[j] = i + gap = 5; \tikz\node[coordinate] (n1) {};
			\end{description}

			\begin{table}[!htb]
				\centering
					\begin{tabular}{|l|c|c|c|c|c|c|c|c|c|c|}
						\hline
						\rowcolor{chameleongreen3}Posição     & \tikz\node(t2){0}; & 1 & 2 & 3 & 4 & \tikz\node(t1){5}; & 6 & 7 & 8 & 9\\ \hline
						\rowcolor{chameleongreen2}\cellcolor{chameleongreen3}Valor       & \tikz\node(n3){2}; & 4 & 6 & 0 & 1 & \tikz\node(t3){7}; & 5 & 3 & 8 & 9\\ \hline
					\end{tabular}
			\end{table}
			\begin{tikzpicture}[overlay]
				\path[->] (n1) edge [out=0, in=90] (t1);
				\path[->] (n2) edge [out=0, in=90] (t2);
				\path[->] (n3) edge [out=-90, in=-90] node[below] {\footnotesize{menor}} (t3);
			\end{tikzpicture}
		\end{frame}

		\begin{frame}
		\frametitle{\secname}
		\framesubtitle{\subsecname}
				
			\begin{description}
				\item[gap] = 5;
				\item[swap] = false;
				\item[i] = 1; \tikz\node[coordinate] (n2) {};
				\item[j] = i + gap = 6; \tikz\node[coordinate] (n1) {};
			\end{description}

			\begin{table}[!htb]
				\centering
					\begin{tabular}{|l|c|c|c|c|c|c|c|c|c|c|}
						\hline
						\rowcolor{chameleongreen3}Posição     & 0 & \tikz\node(t2){1}; & 2 & 3 & 4 & 5 & \tikz\node(t1){6}; & 7 & 8 & 9\\ \hline
						\rowcolor{chameleongreen2}\cellcolor{chameleongreen3}Valor       & 2 & \tikz\node(n3){4}; & 6 & 0 & 1 & 7 & \tikz\node(t3){5}; & 3 & 8 & 9\\ \hline
					\end{tabular}
			\end{table}
			\begin{tikzpicture}[overlay]
				\path[->] (n1) edge [out=0, in=90] (t1);
				\path[->] (n2) edge [out=0, in=90] (t2);
				\path[->] (n3) edge [out=-90, in=-90] node[below] {\footnotesize{menor}} (t3);
			\end{tikzpicture}
		\end{frame}

		\begin{frame}
		\frametitle{\secname}
		\framesubtitle{\subsecname}
				
			\begin{description}
				\item[gap] = 5; 
				\item[swap] = false;
				\item[i] = 2; \tikz\node[coordinate] (n2) {};
				\item[j] = i + gap = 7; \tikz\node[coordinate] (n1) {};
			\end{description}

			\begin{table}[!htb]
				\centering
					\begin{tabular}{|l|c|c|c|c|c|c|c|c|c|c|}
						\hline
						\rowcolor{chameleongreen3}Posição     & 0 & 1 & \tikz\node(t2){2}; & 3 & 4 & 5 & 6 & \tikz\node(t1){7}; & 8 & 9\\ \hline
						\rowcolor{chameleongreen2}\cellcolor{chameleongreen3}Valor       & 2 & 4 & \tikz\node(n3){6}; & 0 & 1 & 7 & 5 & \tikz\node(t3){3}; & 8 & 9\\ \hline
					\end{tabular}
			\end{table}
			\begin{tikzpicture}[overlay]
				\path[->] (n1) edge [out=0, in=90] (t1);
				\path[->] (n2) edge [out=0, in=90] (t2);
				\path[<->,red] (n3) edge [out=-90, in=-90] node[below] {\footnotesize{troca}} (t3);
			\end{tikzpicture}
		\end{frame}

		\begin{frame}
		\frametitle{\secname}
		\framesubtitle{\subsecname}
				
			\begin{description}
				\item[gap] = 5; 
				\item[swap] = true;
				\item[i] = 2; \tikz\node[coordinate] (n2) {};
				\item[j] = i + gap = 7; \tikz\node[coordinate] (n1) {};
			\end{description}

			\begin{table}[!htb]
				\centering
					\begin{tabular}{|l|c|c|c|c|c|c|c|c|c|c|}
						\hline
						\rowcolor{chameleongreen3}Posição     & 0 & 1 & \tikz\node(t2){2}; & 3 & 4 & 5 & 6 & \tikz\node(t1){7}; & 8 & 9\\ \hline
						\rowcolor{chameleongreen2}\cellcolor{chameleongreen3}Valor       & 2 & 4 & \tikz\node(n3){3}; & 0 & 1 & 7 & 5 & \tikz\node(t3){6}; & 8 & 9\\ \hline
					\end{tabular}
			\end{table}
			\begin{tikzpicture}[overlay]
				\path[->] (n1) edge [out=0, in=90] (t1);
				\path[->] (n2) edge [out=0, in=90] (t2);
				\path[<->,red] (n3) edge [out=-90, in=-90] node[below] {\footnotesize{troca}} (t3);
			\end{tikzpicture}
		\end{frame}

		\begin{frame}
		\frametitle{\secname}
		\framesubtitle{\subsecname}
				
			\begin{description}
				\item[gap] = 5;
				\item[swap] = true;
				\item[i] = 3; \tikz\node[coordinate] (n2) {};
				\item[j] = i + gap = 8; \tikz\node[coordinate] (n1) {};
			\end{description}

			\begin{table}[!htb]
				\centering
					\begin{tabular}{|l|c|c|c|c|c|c|c|c|c|c|}
						\hline
						\rowcolor{chameleongreen3}Posição     & 0 & 1 & 2 & \tikz\node(t2){3}; & 4 & 5 & 6 & 7 & \tikz\node(t1){8}; & 9\\ \hline
						\rowcolor{chameleongreen2}\cellcolor{chameleongreen3}Valor       & 2 & 4 & 3 & \tikz\node(n3){0}; & 1 & 7 & 5 & 6 & \tikz\node(t3){8}; & 9\\ \hline
					\end{tabular}
			\end{table}
			\begin{tikzpicture}[overlay]
				\path[->] (n1) edge [out=0, in=90] (t1);
				\path[->] (n2) edge [out=0, in=90] (t2);
				\path[->] (n3) edge [out=-90, in=-90] node[below] {\footnotesize{menor}} (t3);
			\end{tikzpicture}
		\end{frame}

		\begin{frame}
		\frametitle{\secname}
		\framesubtitle{\subsecname}
				
			\begin{description}
				\item[gap] = 5;
				\item[swap] = true;
				\item[i] = 4; \tikz\node[coordinate] (n2) {};
				\item[j] = i + gap = 9; \tikz\node[coordinate] (n1) {};
			\end{description}

			\begin{table}[!htb]
				\centering
					\begin{tabular}{|l|c|c|c|c|c|c|c|c|c|c|}
						\hline
						\rowcolor{chameleongreen3}Posição     & 0 & 1 & 2 & 3 & \tikz\node(t2){4}; & 5 & 6 & 7 & 8 & \tikz\node(t1){9};\\ \hline
						\rowcolor{chameleongreen2}\cellcolor{chameleongreen3}Valor       & 2 & 4 & 3 & 0 & \tikz\node(n3){1}; & 7 & 5 & 6 & 8 & \tikz\node(t3){9};\\ \hline
					\end{tabular}
			\end{table}
			\begin{tikzpicture}[overlay]
				\path[->] (n1) edge [out=0, in=90] (t1);
				\path[->] (n2) edge [out=0, in=90] (t2);
				\path[->] (n3) edge [out=-90, in=-90] node[below] {\footnotesize{menor}} (t3);
			\end{tikzpicture}
		\end{frame}

		\begin{frame}
		\frametitle{\secname}
		\framesubtitle{\subsecname}
				
			\begin{description}
				\item[gap] = gap/FE = $^5/_{1.3}$ = 3;
				\item[swap] = false;
				\item[i] = 0; \tikz\node[coordinate] (n2) {};
				\item[j] = i + gap = 3; \tikz\node[coordinate] (n1) {};
			\end{description}

			\begin{table}[!htb]
				\centering
					\begin{tabular}{|l|c|c|c|c|c|c|c|c|c|c|}
						\hline
						\rowcolor{chameleongreen3}Posição     & \tikz\node(t2){0}; & 1 & 2 & \tikz\node(t1){3}; & 4 & 5 & 6 & 7 & 8 & 9\\ \hline
						\rowcolor{chameleongreen2}\cellcolor{chameleongreen3}Valor       & \tikz\node(n3){2}; & 4 & 3 & \tikz\node(t3){0}; & 1 & 7 & 5 & 6 & 8 & 9\\ \hline
					\end{tabular}
			\end{table}
			\begin{tikzpicture}[overlay]
				\path[->] (n1) edge [out=0, in=90] (t1);
				\path[->] (n2) edge [out=0, in=90] (t2);
				\path[<->,red] (n3) edge [out=-90, in=-90] node[below] {\footnotesize{troca}} (t3);
			\end{tikzpicture}
		\end{frame}

		\begin{frame}
		\frametitle{\secname}
		\framesubtitle{\subsecname}
				
			\begin{description}
				\item[gap] = 3;
				\item[swap] = true;
				\item[i] = 0; \tikz\node[coordinate] (n2) {};
				\item[j] = i + gap = 3; \tikz\node[coordinate] (n1) {};
			\end{description}

			\begin{table}[!htb]
				\centering
					\begin{tabular}{|l|c|c|c|c|c|c|c|c|c|c|}
						\hline
						\rowcolor{chameleongreen3}Posição     & \tikz\node(t2){0}; & 1 & 2 & \tikz\node(t1){3}; & 4 & 5 & 6 & 7 & 8 & 9\\ \hline
						\rowcolor{chameleongreen2}\cellcolor{chameleongreen3}Valor       & \tikz\node(n3){0}; & 4 & 3 & \tikz\node(t3){2}; & 1 & 7 & 5 & 6 & 8 & 9\\ \hline
					\end{tabular}
			\end{table}
			\begin{tikzpicture}[overlay]
				\path[->] (n1) edge [out=0, in=90] (t1);
				\path[->] (n2) edge [out=0, in=90] (t2);
				\path[<->,red] (n3) edge [out=-90, in=-90] node[below] {\footnotesize{troca}} (t3);
			\end{tikzpicture}
		\end{frame}

		\begin{frame}
		\frametitle{\secname}
		\framesubtitle{\subsecname}
				
			\begin{description}
				\item[gap] = 3;
				\item[swap] = true; 
				\item[i] = 1; \tikz\node[coordinate] (n2) {};
				\item[j] = i + gap = 4; \tikz\node[coordinate] (n1) {};
			\end{description}

			\begin{table}[!htb]
				\centering
					\begin{tabular}{|l|c|c|c|c|c|c|c|c|c|c|}
						\hline
						\rowcolor{chameleongreen3}Posição     & 0 & \tikz\node(t2){1}; & 2 & 3 & \tikz\node(t1){4}; & 5 & 6 & 7 & 8 & 9\\ \hline
						\rowcolor{chameleongreen2}\cellcolor{chameleongreen3}Valor       & 0 & \tikz\node(n3){4}; & 3 & 2 & \tikz\node(t3){1}; & 7 & 5 & 6 & 8 & 9\\ \hline
					\end{tabular}
			\end{table}
			\begin{tikzpicture}[overlay]
				\path[->] (n1) edge [out=0, in=90] (t1);
				\path[->] (n2) edge [out=0, in=90] (t2);
				\path[<->,red] (n3) edge [out=-90, in=-90] node[below] {\footnotesize{troca}} (t3);
			\end{tikzpicture}
		\end{frame}

		\begin{frame}
		\frametitle{\secname}
		\framesubtitle{\subsecname}
				
			\begin{description}
				\item[gap] = 3;
				\item[swap] = true; 
				\item[i] = 1; \tikz\node[coordinate] (n2) {};
				\item[j] = i + gap = 4; \tikz\node[coordinate] (n1) {};
			\end{description}

			\begin{table}[!htb]
				\centering
					\begin{tabular}{|l|c|c|c|c|c|c|c|c|c|c|}
						\hline
						\rowcolor{chameleongreen3}Posição     & 0 & \tikz\node(t2){1}; & 2 & 3 & \tikz\node(t1){4}; & 5 & 6 & 7 & 8 & 9\\ \hline
						\rowcolor{chameleongreen2}\cellcolor{chameleongreen3}Valor       & 0 & \tikz\node(n3){1}; & 3 & 2 & \tikz\node(t3){4}; & 7 & 5 & 6 & 8 & 9\\ \hline
					\end{tabular}
			\end{table}
			\begin{tikzpicture}[overlay]
				\path[->] (n1) edge [out=0, in=90] (t1);
				\path[->] (n2) edge [out=0, in=90] (t2);
				\path[<->,red] (n3) edge [out=-90, in=-90] node[below] {\footnotesize{troca}} (t3);
			\end{tikzpicture}
		\end{frame}

		\begin{frame}
		\frametitle{\secname}
		\framesubtitle{\subsecname}
				
			\begin{description}
				\item[gap] = 3;
				\item[swap] = true;
				\item[i] = 2; \tikz\node[coordinate] (n2) {};
				\item[j] = i + gap = 5; \tikz\node[coordinate] (n1) {};
			\end{description}

			\begin{table}[!htb]
				\centering
					\begin{tabular}{|l|c|c|c|c|c|c|c|c|c|c|}
						\hline
						\rowcolor{chameleongreen3}Posição     & 0 & 1 & \tikz\node(t2){2}; & 3 & 4 & \tikz\node(t1){5}; & 6 & 7 & 8 & 9\\ \hline
						\rowcolor{chameleongreen2}\cellcolor{chameleongreen3}Valor       & 0 & 1 & \tikz\node(n3){3}; & 2 & 4 & \tikz\node(t3){7}; & 5 & 6 & 8 & 9\\ \hline
					\end{tabular}
			\end{table}
			\begin{tikzpicture}[overlay]
				\path[->] (n1) edge [out=0, in=90] (t1);
				\path[->] (n2) edge [out=0, in=90] (t2);
				\path[->] (n3) edge [out=-90, in=-90] node[below] {\footnotesize{menor}} (t3);
			\end{tikzpicture}
		\end{frame}
		
		\begin{frame}
		\frametitle{\secname}
		\framesubtitle{\subsecname}
				
			\begin{description}
				\item[gap] = 3;
				\item[swap] = true;
				\item[i] = 3; \tikz\node[coordinate] (n2) {};
				\item[j] = i + gap = 6; \tikz\node[coordinate] (n1) {};
			\end{description}

			\begin{table}[!htb]
				\centering
					\begin{tabular}{|l|c|c|c|c|c|c|c|c|c|c|}
						\hline
						\rowcolor{chameleongreen3}Posição     & 0 & 1 & 2 & \tikz\node(t2){3}; & 4 & 5 & \tikz\node(t1){6}; & 7 & 8 & 9\\ \hline
						\rowcolor{chameleongreen2}\cellcolor{chameleongreen3}Valor       & 0 & 1 & 3 & \tikz\node(n3){2}; & 4 & 7 & \tikz\node(t3){5}; & 6 & 8 & 9\\ \hline
					\end{tabular}
			\end{table}
			\begin{tikzpicture}[overlay]
				\path[->] (n1) edge [out=0, in=90] (t1);
				\path[->] (n2) edge [out=0, in=90] (t2);
				\path[->] (n3) edge [out=-90, in=-90] node[below] {\footnotesize{menor}} (t3);
			\end{tikzpicture}
		\end{frame}

		\begin{frame}
		\frametitle{\secname}
		\framesubtitle{\subsecname}
				
			\begin{description}
				\item[gap] = 3;
				\item[swap] = true; 
				\item[i] = 4; \tikz\node[coordinate] (n2) {};
				\item[j] = i + gap = 7; \tikz\node[coordinate] (n1) {};
			\end{description}

			\begin{table}[!htb]
				\centering
					\begin{tabular}{|l|c|c|c|c|c|c|c|c|c|c|}
						\hline
						\rowcolor{chameleongreen3}Posição     & 0 & 1 & 2 & 3 & \tikz\node(t2){4}; & 5 & 6 & \tikz\node(t1){7}; & 8 & 9\\ \hline
						\rowcolor{chameleongreen2}\cellcolor{chameleongreen3}Valor       & 0 & 1 & 3 & 2 & \tikz\node(n3){4}; & 7 & 5 & \tikz\node(t3){6}; & 8 & 9\\ \hline
					\end{tabular}
			\end{table}
			\begin{tikzpicture}[overlay]
				\path[->] (n1) edge [out=0, in=90] (t1);
				\path[->] (n2) edge [out=0, in=90] (t2);
				\path[->] (n3) edge [out=-90, in=-90] node[below] {\footnotesize{menor}} (t3);
			\end{tikzpicture}
		\end{frame}

		\begin{frame}
		\frametitle{\secname}
		\framesubtitle{\subsecname}
				
			\begin{description}
				\item[gap] = 3; 
				\item[swap] = true;
				\item[i] = 5; \tikz\node[coordinate] (n2) {};
				\item[j] = i + gap = 8; \tikz\node[coordinate] (n1) {};
			\end{description}

			\begin{table}[!htb]
				\centering
					\begin{tabular}{|l|c|c|c|c|c|c|c|c|c|c|}
						\hline
						\rowcolor{chameleongreen3}Posição     & 0 & 1 & 2 & 3 & 4 & \tikz\node(t2){5}; & 6 & 7 & \tikz\node(t1){8}; & 9\\ \hline
						\rowcolor{chameleongreen2}\cellcolor{chameleongreen3}Valor       & 0 & 1 & 3 & 2 & 4 & \tikz\node(n3){7}; & 5 & 6 & \tikz\node(t3){8}; & 9\\ \hline
					\end{tabular}
			\end{table}
			\begin{tikzpicture}[overlay]
				\path[->] (n1) edge [out=0, in=90] (t1);
				\path[->] (n2) edge [out=0, in=90] (t2);
				\path[->] (n3) edge [out=-90, in=-90] node[below] {\footnotesize{menor}} (t3);
			\end{tikzpicture}
		\end{frame}

		\begin{frame}
		\frametitle{\secname}
		\framesubtitle{\subsecname}
				
			\begin{description}
				\item[gap] = 3;
				\item[swap] = true;
				\item[i] = 6; \tikz\node[coordinate] (n2) {};
				\item[j] = i + gap = 9; \tikz\node[coordinate] (n1) {};
			\end{description}

			\begin{table}[!htb]
				\centering
					\begin{tabular}{|l|c|c|c|c|c|c|c|c|c|c|}
						\hline
						\rowcolor{chameleongreen3}Posição     & 0 & 1 & 2 & 3 & 4 & 5 & \tikz\node(t2){6}; & 7 & 8 & \tikz\node(t1){9};\\ \hline
						\rowcolor{chameleongreen2}\cellcolor{chameleongreen3}Valor       & 0 & 1 & 3 & 2 & 4 & 7 & \tikz\node(n3){5}; & 6 & 8 & \tikz\node(t3){9};\\ \hline
					\end{tabular}
			\end{table}
			\begin{tikzpicture}[overlay]
				\path[->] (n1) edge [out=0, in=90] (t1);
				\path[->] (n2) edge [out=0, in=90] (t2);
				\path[->] (n3) edge [out=-90, in=-90] node[below] {\footnotesize{menor}} (t3);
			\end{tikzpicture}
		\end{frame}

		\begin{frame}
		\frametitle{\secname}
		\framesubtitle{\subsecname}
				
			\begin{description}
				\item[gap] = gap/FE = $^3/_{1.3}$ = 2;
				\item[swap] = false;
				\item[i] = 0; \tikz\node[coordinate] (n2) {};
				\item[j] = i + gap = 2; \tikz\node[coordinate] (n1) {};
			\end{description}

			\begin{table}[!htb]
				\centering
					\begin{tabular}{|l|c|c|c|c|c|c|c|c|c|c|}
						\hline
						\rowcolor{chameleongreen3}Posição     & \tikz\node(t2){0}; & 1 & \tikz\node(t1){2}; & 3 & 4 & 5 & 6 & 7 & 8 & 9\\ \hline
						\rowcolor{chameleongreen2}\cellcolor{chameleongreen3}Valor       & \tikz\node(n3){0}; & 1 & \tikz\node(t3){3}; & 2 & 4 & 7 & 5 & 6 & 8 & 9\\ \hline
					\end{tabular}
			\end{table}
			\begin{tikzpicture}[overlay]
				\path[->] (n1) edge [out=0, in=90] (t1);
				\path[->] (n2) edge [out=0, in=90] (t2);
				\path[->] (n3) edge [out=-90, in=-90] node[below] {\footnotesize{menor}} (t3);
			\end{tikzpicture}
		\end{frame}

		\begin{frame}
		\frametitle{\secname}
		\framesubtitle{\subsecname}
				
			\begin{description}
				\item[gap] =  2; 
				\item[swap] = false;
				\item[i] = 1; \tikz\node[coordinate] (n2) {};
				\item[j] = i + gap = 3; \tikz\node[coordinate] (n1) {};
			\end{description}

			\begin{table}[!htb]
				\centering
					\begin{tabular}{|l|c|c|c|c|c|c|c|c|c|c|}
						\hline
						\rowcolor{chameleongreen3}Posição     & 0 & \tikz\node(t2){1}; & 2 & \tikz\node(t1){3}; & 4 & 5 & 6 & 7 & 8 & 9\\ \hline
						\rowcolor{chameleongreen2}\cellcolor{chameleongreen3}Valor       & 0 & \tikz\node(n3){1}; & 3 & \tikz\node(t3){2}; & 4 & 7 & 5 & 6 & 8 & 9\\ \hline
					\end{tabular}
			\end{table}
			\begin{tikzpicture}[overlay]
				\path[->] (n1) edge [out=0, in=90] (t1);
				\path[->] (n2) edge [out=0, in=90] (t2);
				\path[->] (n3) edge [out=-90, in=-90] node[below] {\footnotesize{menor}} (t3);
			\end{tikzpicture}
		\end{frame}

		\begin{frame}
		\frametitle{\secname}
		\framesubtitle{\subsecname}
				
			\begin{description}
				\item[gap] =  2; 
				\item[swap] = false;
				\item[i] = 2; \tikz\node[coordinate] (n2) {};
				\item[j] = i + gap = 4; \tikz\node[coordinate] (n1) {};
			\end{description}

			\begin{table}[!htb]
				\centering
					\begin{tabular}{|l|c|c|c|c|c|c|c|c|c|c|}
						\hline
						\rowcolor{chameleongreen3}Posição     & 0 & 1 & \tikz\node(t2){2}; & 3 & \tikz\node(t1){4}; & 5 & 6 & 7 & 8 & 9\\ \hline
						\rowcolor{chameleongreen2}\cellcolor{chameleongreen3}Valor       & 0 & 1 & \tikz\node(n3){3}; & 2 & \tikz\node(t3){4}; & 7 & 5 & 6 & 8 & 9\\ \hline
					\end{tabular}
			\end{table}
			\begin{tikzpicture}[overlay]
				\path[->] (n1) edge [out=0, in=90] (t1);
				\path[->] (n2) edge [out=0, in=90] (t2);
				\path[->] (n3) edge [out=-90, in=-90] node[below] {\footnotesize{menor}} (t3);
			\end{tikzpicture}
		\end{frame}

		\begin{frame}
		\frametitle{\secname}
		\framesubtitle{\subsecname}
				
			\begin{description}
				\item[gap] =  2; 
				\item[swap] = false;
				\item[i] = 3; \tikz\node[coordinate] (n2) {};
				\item[j] = i + gap = 5; \tikz\node[coordinate] (n1) {};
			\end{description}

			\begin{table}[!htb]
				\centering
					\begin{tabular}{|l|c|c|c|c|c|c|c|c|c|c|}
						\hline
						\rowcolor{chameleongreen3}Posição     & 0 & 1 & 2 & \tikz\node(t2){3}; & 4 & \tikz\node(t1){5}; & 6 & 7 & 8 & 9\\ \hline
						\rowcolor{chameleongreen2}\cellcolor{chameleongreen3}Valor       & 0 & 1 & 3 & \tikz\node(n3){2}; & 4 & \tikz\node(t3){7}; & 5 & 6 & 8 & 9\\ \hline
					\end{tabular}
			\end{table}
			\begin{tikzpicture}[overlay]
				\path[->] (n1) edge [out=0, in=90] (t1);
				\path[->] (n2) edge [out=0, in=90] (t2);
				\path[->] (n3) edge [out=-90, in=-90] node[below] {\footnotesize{menor}} (t3);
			\end{tikzpicture}
		\end{frame}

		\begin{frame}
		\frametitle{\secname}
		\framesubtitle{\subsecname}
				
			\begin{description}
				\item[gap] =  2;
				\item[swap] = false;
				\item[i] = 4; \tikz\node[coordinate] (n2) {};
				\item[j] = i + gap = 6; \tikz\node[coordinate] (n1) {};
			\end{description}

			\begin{table}[!htb]
				\centering
					\begin{tabular}{|l|c|c|c|c|c|c|c|c|c|c|}
						\hline
						\rowcolor{chameleongreen3}Posição     & 0 & 1 & 2 & 3 & \tikz\node(t2){4}; & 5 & \tikz\node(t1){6}; & 7 & 8 & 9\\ \hline
						\rowcolor{chameleongreen2}\cellcolor{chameleongreen3}Valor       & 0 & 1 & 3 & 2 & \tikz\node(n3){4}; & 7 & \tikz\node(t3){5}; & 6 & 8 & 9\\ \hline
					\end{tabular}
			\end{table}
			\begin{tikzpicture}[overlay]
				\path[->] (n1) edge [out=0, in=90] (t1);
				\path[->] (n2) edge [out=0, in=90] (t2);
				\path[->] (n3) edge [out=-90, in=-90] node[below] {\footnotesize{menor}} (t3);
			\end{tikzpicture}
		\end{frame}

		\begin{frame}
		\frametitle{\secname}
		\framesubtitle{\subsecname}
				
			\begin{description}
				\item[gap] =  2; 
				\item[swap] = false;
				\item[i] = 5; \tikz\node[coordinate] (n2) {};
				\item[j] = i + gap = 7; \tikz\node[coordinate] (n1) {};
			\end{description}

			\begin{table}[!htb]
				\centering
					\begin{tabular}{|l|c|c|c|c|c|c|c|c|c|c|}
						\hline
						\rowcolor{chameleongreen3}Posição     & 0 & 1 & 2 & 3 & 4 & \tikz\node(t2){5}; & 6 & \tikz\node(t1){7}; & 8 & 9\\ \hline
						\rowcolor{chameleongreen2}\cellcolor{chameleongreen3}Valor       & 0 & 1 & 3 & 2 & 4 & \tikz\node(n3){7}; & 5 & \tikz\node(t3){6}; & 8 & 9\\ \hline
					\end{tabular}
			\end{table}
			\begin{tikzpicture}[overlay]
				\path[->] (n1) edge [out=0, in=90] (t1);
				\path[->] (n2) edge [out=0, in=90] (t2);
				\path[<->,red] (n3) edge [out=-90, in=-90] node[below] {\footnotesize{troca}} (t3);
			\end{tikzpicture}
		\end{frame}

		\begin{frame}
		\frametitle{\secname}
		\framesubtitle{\subsecname}
				
			\begin{description}
				\item[gap] =  2; 
				\item[swap] = true;
				\item[i] = 5; \tikz\node[coordinate] (n2) {};
				\item[j] = i + gap = 7; \tikz\node[coordinate] (n1) {};
			\end{description}

			\begin{table}[!htb]
				\centering
					\begin{tabular}{|l|c|c|c|c|c|c|c|c|c|c|}
						\hline
						\rowcolor{chameleongreen3}Posição     & 0 & 1 & 2 & 3 & 4 & \tikz\node(t2){5}; & 6 & \tikz\node(t1){7}; & 8 & 9\\ \hline
						\rowcolor{chameleongreen2}\cellcolor{chameleongreen3}Valor       & 0 & 1 & 3 & 2 & 4 & \tikz\node(n3){6}; & 5 & \tikz\node(t3){7}; & 8 & 9\\ \hline
					\end{tabular}
			\end{table}
			\begin{tikzpicture}[overlay]
				\path[->] (n1) edge [out=0, in=90] (t1);
				\path[->] (n2) edge [out=0, in=90] (t2);
				\path[<->,red] (n3) edge [out=-90, in=-90] node[below] {\footnotesize{troca}} (t3);
			\end{tikzpicture}
		\end{frame}

		\begin{frame}
		\frametitle{\secname}
		\framesubtitle{\subsecname}
				
			\begin{description}
				\item[gap] =  2; 
				\item[swap] = true;
				\item[i] = 6; \tikz\node[coordinate] (n2) {};
				\item[j] = i + gap = 8; \tikz\node[coordinate] (n1) {};
			\end{description}

			\begin{table}[!htb]
				\centering
					\begin{tabular}{|l|c|c|c|c|c|c|c|c|c|c|}
						\hline
						\rowcolor{chameleongreen3}Posição     & 0 & 1 & 2 & 3 & 4 & 5 & \tikz\node(t2){6}; & 7 & \tikz\node(t1){8}; & 9\\ \hline
						\rowcolor{chameleongreen2}\cellcolor{chameleongreen3}Valor       & 0 & 1 & 3 & 2 & 4 & 6 & \tikz\node(n3){5}; & 7 & \tikz\node(t3){8}; & 9\\ \hline
					\end{tabular}
			\end{table}
			\begin{tikzpicture}[overlay]
				\path[->] (n1) edge [out=0, in=90] (t1);
				\path[->] (n2) edge [out=0, in=90] (t2);
				\path[->] (n3) edge [out=-90, in=-90] node[below] {\footnotesize{menor}} (t3);
			\end{tikzpicture}
		\end{frame}

		\begin{frame}
		\frametitle{\secname}
		\framesubtitle{\subsecname}
				
			\begin{description}
				\item[gap] =  2; 
				\item[swap] = true;
				\item[i] = 7; \tikz\node[coordinate] (n2) {};
				\item[j] = i + gap = 9; \tikz\node[coordinate] (n1) {};
			\end{description}

			\begin{table}[!htb]
				\centering
					\begin{tabular}{|l|c|c|c|c|c|c|c|c|c|c|}
						\hline
						\rowcolor{chameleongreen3}Posição     & 0 & 1 & 2 & 3 & 4 & 5 & 6 & \tikz\node(t2){7}; & 8 & \tikz\node(t1){9};\\ \hline
						\rowcolor{chameleongreen2}\cellcolor{chameleongreen3}Valor       & 0 & 1 & 3 & 2 & 4 & 6 & 5 & \tikz\node(n3){7}; & 8 & \tikz\node(t3){9};\\ \hline
					\end{tabular}
			\end{table}
			\begin{tikzpicture}[overlay]
				\path[->] (n1) edge [out=0, in=90] (t1);
				\path[->] (n2) edge [out=0, in=90] (t2);
				\path[->] (n3) edge [out=-90, in=-90] node[below] {\footnotesize{menor}} (t3);
			\end{tikzpicture}
		\end{frame}

		\begin{frame}
		\frametitle{\secname}
		\framesubtitle{\subsecname}
				
			\begin{description}
				\item[gap] =  gap/FE = $^2/_{1.3}$ = 1; 
				\item[swap] = false;
				\item[i] = 0; \tikz\node[coordinate] (n2) {};
				\item[j] = i + gap = 1; \tikz\node[coordinate] (n1) {};
			\end{description}

			\begin{table}[!htb]
				\centering
					\begin{tabular}{|l|c|c|c|c|c|c|c|c|c|c|}
						\hline
						\rowcolor{chameleongreen3}Posição     & \tikz\node(t2){0}; & \tikz\node(t1){1}; & 2 & 3 & 4 & 5 & 6 & 7 & 8 & 9\\ \hline
						\rowcolor{chameleongreen2}\cellcolor{chameleongreen3}Valor       & \tikz\node(n3){0}; & \tikz\node(t3){1}; & 3 & 2 & 4 & 6 & 5 & 7 & 8 & 9\\ \hline
					\end{tabular}
			\end{table}
			\begin{tikzpicture}[overlay]
				\path[->] (n1) edge [out=0, in=90] (t1);
				\path[->] (n2) edge [out=0, in=90] (t2);
				\path[->] (n3) edge [out=-90, in=-90] node[below] {\footnotesize{menor}} (t3);
			\end{tikzpicture}
		\end{frame}

		\begin{frame}
		\frametitle{\secname}
		\framesubtitle{\subsecname}
				
			\begin{description}
				\item[gap] = 1;
				\item[swap] = false;
				\item[i] = 1; \tikz\node[coordinate] (n2) {};
				\item[j] = i + gap = 2; \tikz\node[coordinate] (n1) {};
			\end{description}

			\begin{table}[!htb]
				\centering
					\begin{tabular}{|l|c|c|c|c|c|c|c|c|c|c|}
						\hline
						\rowcolor{chameleongreen3}Posição     & 0 & \tikz\node(t2){1}; & \tikz\node(t1){2}; & 3 & 4 & 5 & 6 & 7 & 8 & 9\\ \hline
						\rowcolor{chameleongreen2}\cellcolor{chameleongreen3}Valor       & 0 & \tikz\node(n3){1}; & \tikz\node(t3){3}; & 2 & 4 & 6 & 5 & 7 & 8 & 9\\ \hline
					\end{tabular}
			\end{table}
			\begin{tikzpicture}[overlay]
				\path[->] (n1) edge [out=0, in=90] (t1);
				\path[->] (n2) edge [out=0, in=90] (t2);
				\path[->] (n3) edge [out=-90, in=-90] node[below] {\footnotesize{menor}} (t3);
			\end{tikzpicture}
		\end{frame}

		\begin{frame}
		\frametitle{\secname}
		\framesubtitle{\subsecname}
				
			\begin{description}
				\item[gap] = 1;
				\item[swap] = false; 
				\item[i] = 2; \tikz\node[coordinate] (n2) {};
				\item[j] = i + gap = 3; \tikz\node[coordinate] (n1) {};
			\end{description}

			\begin{table}[!htb]
				\centering
					\begin{tabular}{|l|c|c|c|c|c|c|c|c|c|c|}
						\hline
						\rowcolor{chameleongreen3}Posição     & 0 & 1 & \tikz\node(t2){2}; & \tikz\node(t1){3}; & 4 & 5 & 6 & 7 & 8 & 9\\ \hline
						\rowcolor{chameleongreen2}\cellcolor{chameleongreen3}Valor       & 0 & 1 & \tikz\node(n3){3}; & \tikz\node(t3){2}; & 4 & 6 & 5 & 7 & 8 & 9\\ \hline
					\end{tabular}
			\end{table}
			\begin{tikzpicture}[overlay]
				\path[->] (n1) edge [out=0, in=90] (t1);
				\path[->] (n2) edge [out=0, in=90] (t2);
				\path[<->,red] (n3) edge [out=-90, in=-90] node[below] {\footnotesize{troca}} (t3);
			\end{tikzpicture}
		\end{frame}

		\begin{frame}
		\frametitle{\secname}
		\framesubtitle{\subsecname}
				
			\begin{description}
				\item[gap] = 1;
				\item[swap] = true; 
				\item[i] = 2; \tikz\node[coordinate] (n2) {};
				\item[j] = i + gap = 3; \tikz\node[coordinate] (n1) {};
			\end{description}

			\begin{table}[!htb]
				\centering
					\begin{tabular}{|l|c|c|c|c|c|c|c|c|c|c|}
						\hline
						\rowcolor{chameleongreen3}Posição     & 0 & 1 & \tikz\node(t2){2}; & \tikz\node(t1){3}; & 4 & 5 & 6 & 7 & 8 & 9\\ \hline
						\rowcolor{chameleongreen2}\cellcolor{chameleongreen3}Valor       & 0 & 1 & \tikz\node(n3){2}; & \tikz\node(t3){3}; & 4 & 6 & 5 & 7 & 8 & 9\\ \hline
					\end{tabular}
			\end{table}
			\begin{tikzpicture}[overlay]
				\path[->] (n1) edge [out=0, in=90] (t1);
				\path[->] (n2) edge [out=0, in=90] (t2);
				\path[<->,red] (n3) edge [out=-90, in=-90] node[below] {\footnotesize{troca}} (t3);
			\end{tikzpicture}
		\end{frame}

		\begin{frame}
		\frametitle{\secname}
		\framesubtitle{\subsecname}
				
			\begin{description}
				\item[gap] = 1; 
				\item[swap] = true; 
				\item[i] = 3; \tikz\node[coordinate] (n2) {};
				\item[j] = i + gap = 4; \tikz\node[coordinate] (n1) {};
			\end{description}

			\begin{table}[!htb]
				\centering
					\begin{tabular}{|l|c|c|c|c|c|c|c|c|c|c|}
						\hline
						\rowcolor{chameleongreen3}Posição     & 0 & 1 & 2 & \tikz\node(t2){3}; & \tikz\node(t1){4}; & 5 & 6 & 7 & 8 & 9\\ \hline
						\rowcolor{chameleongreen2}\cellcolor{chameleongreen3}Valor       & 0 & 1 & 2 & \tikz\node(n3){3}; & \tikz\node(t3){4}; & 6 & 5 & 7 & 8 & 9\\ \hline
					\end{tabular}
			\end{table}
			\begin{tikzpicture}[overlay]
				\path[->] (n1) edge [out=0, in=90] (t1);
				\path[->] (n2) edge [out=0, in=90] (t2);
				\path[->] (n3) edge [out=-90, in=-90] node[below] {\footnotesize{menor}} (t3);
			\end{tikzpicture}
		\end{frame}

		\begin{frame}
		\frametitle{\secname}
		\framesubtitle{\subsecname}
				
			\begin{description}
				\item[gap] = 1; 
				\item[swap] = true; 
				\item[i] = 4; \tikz\node[coordinate] (n2) {};
				\item[j] = i + gap = 5; \tikz\node[coordinate] (n1) {};
			\end{description}

			\begin{table}[!htb]
				\centering
					\begin{tabular}{|l|c|c|c|c|c|c|c|c|c|c|}
						\hline
						\rowcolor{chameleongreen3}Posição     & 0 & 1 & 2 & 3 & \tikz\node(t2){4}; & \tikz\node(t1){5}; & 6 & 7 & 8 & 9\\ \hline
						\rowcolor{chameleongreen2}\cellcolor{chameleongreen3}Valor       & 0 & 1 & 2 & 3 & \tikz\node(n3){4}; & \tikz\node(t3){6}; & 5 & 7 & 8 & 9\\ \hline
					\end{tabular}
			\end{table}
			\begin{tikzpicture}[overlay]
				\path[->] (n1) edge [out=0, in=90] (t1);
				\path[->] (n2) edge [out=0, in=90] (t2);
				\path[->] (n3) edge [out=-90, in=-90] node[below] {\footnotesize{menor}} (t3);
			\end{tikzpicture}
		\end{frame}

		\begin{frame}
		\frametitle{\secname}
		\framesubtitle{\subsecname}
				
			\begin{description}
				\item[gap] = 1; 
				\item[swap] = true; 
				\item[i] = 5; \tikz\node[coordinate] (n2) {};
				\item[j] = i + gap = 6; \tikz\node[coordinate] (n1) {};
			\end{description}

			\begin{table}[!htb]
				\centering
					\begin{tabular}{|l|c|c|c|c|c|c|c|c|c|c|}
						\hline
						\rowcolor{chameleongreen3}Posição     & 0 & 1 & 2 & 3 & 4 & \tikz\node(t2){5}; & \tikz\node(t1){6}; & 7 & 8 & 9\\ \hline
						\rowcolor{chameleongreen2}\cellcolor{chameleongreen3}Valor       & 0 & 1 & 2 & 3 & 4 & \tikz\node(n3){6}; & \tikz\node(t3){5}; & 7 & 8 & 9\\ \hline
					\end{tabular}
			\end{table}
			\begin{tikzpicture}[overlay]
				\path[->] (n1) edge [out=0, in=90] (t1);
				\path[->] (n2) edge [out=0, in=90] (t2);
				\path[<->,red] (n3) edge [out=-90, in=-90] node[below] {\footnotesize{troca}} (t3);
			\end{tikzpicture}
		\end{frame}

		\begin{frame}
		\frametitle{\secname}
		\framesubtitle{\subsecname}
				
			\begin{description}
				\item[gap] = 1; 
				\item[swap] = true; 
				\item[i] = 5; \tikz\node[coordinate] (n2) {};
				\item[j] = i + gap = 6; \tikz\node[coordinate] (n1) {};
			\end{description}

			\begin{table}[!htb]
				\centering
					\begin{tabular}{|l|c|c|c|c|c|c|c|c|c|c|}
						\hline
						\rowcolor{chameleongreen3}Posição     & 0 & 1 & 2 & 3 & 4 & \tikz\node(t2){5}; & \tikz\node(t1){6}; & 7 & 8 & 9\\ \hline
						\rowcolor{chameleongreen2}\cellcolor{chameleongreen3}Valor       & 0 & 1 & 2 & 3 & 4 & \tikz\node(n3){5}; & \tikz\node(t3){6}; & 7 & 8 & 9\\ \hline
					\end{tabular}
			\end{table}
			\begin{tikzpicture}[overlay]
				\path[->] (n1) edge [out=0, in=90] (t1);
				\path[->] (n2) edge [out=0, in=90] (t2);
				\path[<->,red] (n3) edge [out=-90, in=-90] node[below] {\footnotesize{troca}} (t3);
			\end{tikzpicture}
		\end{frame}

		\begin{frame}
		\frametitle{\secname}
		\framesubtitle{\subsecname}
				
			\begin{description}
				\item[gap] = 1; 
				\item[swap] = true; 
				\item[i] = 6; \tikz\node[coordinate] (n2) {};
				\item[j] = i + gap = 7; \tikz\node[coordinate] (n1) {};
			\end{description}

			\begin{table}[!htb]
				\centering
					\begin{tabular}{|l|c|c|c|c|c|c|c|c|c|c|}
						\hline
						\rowcolor{chameleongreen3}Posição     & 0 & 1 & 2 & 3 & 4 & 5 & \tikz\node(t2){6}; & \tikz\node(t1){7}; & 8 & 9\\ \hline
						\rowcolor{chameleongreen2}\cellcolor{chameleongreen3}Valor       & 0 & 1 & 2 & 3 & 4 & 5 & \tikz\node(n3){6}; & \tikz\node(t3){7}; & 8 & 9\\ \hline
					\end{tabular}
			\end{table}
			\begin{tikzpicture}[overlay]
				\path[->] (n1) edge [out=0, in=90] (t1);
				\path[->] (n2) edge [out=0, in=90] (t2);
				\path[->] (n3) edge [out=-90, in=-90] node[below] {\footnotesize{menor}} (t3);
			\end{tikzpicture}
		\end{frame}

		\begin{frame}
		\frametitle{\secname}
		\framesubtitle{\subsecname}
				
			\begin{description}
				\item[gap] = 1; 
				\item[swap] = true; 
				\item[i] = 7; \tikz\node[coordinate] (n2) {};
				\item[j] = i + gap = 8; \tikz\node[coordinate] (n1) {};
			\end{description}

			\begin{table}[!htb]
				\centering
					\begin{tabular}{|l|c|c|c|c|c|c|c|c|c|c|}
						\hline
						\rowcolor{chameleongreen3}Posição     & 0 & 1 & 2 & 3 & 4 & 5 & 6 & \tikz\node(t2){7}; & \tikz\node(t1){8}; & 9\\ \hline
						\rowcolor{chameleongreen2}\cellcolor{chameleongreen3}Valor       & 0 & 1 & 2 & 3 & 4 & 5 & 6 & \tikz\node(n3){7}; & \tikz\node(t3){8}; & 9\\ \hline
					\end{tabular}
			\end{table}
			\begin{tikzpicture}[overlay]
				\path[->] (n1) edge [out=0, in=90] (t1);
				\path[->] (n2) edge [out=0, in=90] (t2);
				\path[->] (n3) edge [out=-90, in=-90] node[below] {\footnotesize{menor}} (t3);
			\end{tikzpicture}
		\end{frame}

		\begin{frame}
		\frametitle{\secname}
		\framesubtitle{\subsecname}
				
			\begin{description}
				\item[gap] = 1; 
				\item[swap] = true; 
				\item[i] = 8; \tikz\node[coordinate] (n2) {};
				\item[j] = i + gap = 9; \tikz\node[coordinate] (n1) {};
			\end{description}

			\begin{table}[!htb]
				\centering
					\begin{tabular}{|l|c|c|c|c|c|c|c|c|c|c|}
						\hline
						\rowcolor{chameleongreen3}Posição     & 0 & 1 & 2 & 3 & 4 & 5 & 6 & 7 & \tikz\node(t2){8}; & \tikz\node(t1){9};\\ \hline
						\rowcolor{chameleongreen2}\cellcolor{chameleongreen3}Valor       & 0 & 1 & 2 & 3 & 4 & 5 & 6 & 7 & \tikz\node(n3){8}; & \tikz\node(t3){9};\\ \hline
					\end{tabular}
			\end{table}
			\begin{tikzpicture}[overlay]
				\path[->] (n1) edge [out=0, in=90] (t1);
				\path[->] (n2) edge [out=0, in=90] (t2);
				\path[->] (n3) edge [out=-90, in=-90] node[below] {\footnotesize{menor}} (t3);
			\end{tikzpicture}
		\end{frame}

		\begin{frame}
		\frametitle{\secname}
		\framesubtitle{\subsecname}
				
			\begin{description}
				\item[gap] = 1; 
				\item[swap] = false; 
				\item[i] = 0; \tikz\node[coordinate] (n2) {};
				\item[j] = i + gap = 1; \tikz\node[coordinate] (n1) {};
			\end{description}

			\begin{table}[!htb]
				\centering
					\begin{tabular}{|l|c|c|c|c|c|c|c|c|c|c|}
						\hline
						\rowcolor{chameleongreen3}Posição     & \tikz\node(t2){0}; & \tikz\node(t1){1}; & 2 & 3 & 4 & 5 & 6 & 7 & 8 & 9\\ \hline
						\rowcolor{chameleongreen2}\cellcolor{chameleongreen3}Valor       & \tikz\node(n3){0}; & \tikz\node(t3){1}; & 2 & 2 & 4 & 5 & 6 & 7 & 8 & 9\\ \hline
					\end{tabular}
			\end{table}
			\begin{tikzpicture}[overlay]
				\path[->] (n1) edge [out=0, in=90] (t1);
				\path[->] (n2) edge [out=0, in=90] (t2);
				\path[->] (n3) edge [out=-90, in=-90] node[below] {\footnotesize{menor}} (t3);
			\end{tikzpicture}
		\end{frame}

		\begin{frame}
		\frametitle{\secname}
		\framesubtitle{\subsecname}
				
			\begin{description}
				\item[gap] = 1;
				\item[swap] = false;  
				\item[i] = 1; \tikz\node[coordinate] (n2) {};
				\item[j] = i + gap = 2; \tikz\node[coordinate] (n1) {};
			\end{description}

			\begin{table}[!htb]
				\centering
					\begin{tabular}{|l|c|c|c|c|c|c|c|c|c|c|}
						\hline
						\rowcolor{chameleongreen3}Posição     & 0 & \tikz\node(t2){1}; & \tikz\node(t1){2}; & 3 & 4 & 5 & 6 & 7 & 8 & 9\\ \hline
						\rowcolor{chameleongreen2}\cellcolor{chameleongreen3}Valor       & 0 & \tikz\node(n3){1}; & \tikz\node(t3){2}; & 3 & 4 & 5 & 6 & 7 & 8 & 9\\ \hline
					\end{tabular}
			\end{table}
			\begin{tikzpicture}[overlay]
				\path[->] (n1) edge [out=0, in=90] (t1);
				\path[->] (n2) edge [out=0, in=90] (t2);
				\path[->] (n3) edge [out=-90, in=-90] node[below] {\footnotesize{menor}} (t3);
			\end{tikzpicture}
		\end{frame}

		\begin{frame}
		\frametitle{\secname}
		\framesubtitle{\subsecname}
				
			\begin{description}
				\item[gap] = 1; 
				\item[swap] = false; 
				\item[i] = 2; \tikz\node[coordinate] (n2) {};
				\item[j] = i + gap = 3; \tikz\node[coordinate] (n1) {};
			\end{description}

			\begin{table}[!htb]
				\centering
					\begin{tabular}{|l|c|c|c|c|c|c|c|c|c|c|}
						\hline
						\rowcolor{chameleongreen3}Posição     & 0 & 1 & \tikz\node(t2){2}; & \tikz\node(t1){3}; & 4 & 5 & 6 & 7 & 8 & 9\\ \hline
						\rowcolor{chameleongreen2}\cellcolor{chameleongreen3}Valor       & 0 & 1 & \tikz\node(n3){2}; & \tikz\node(t3){3}; & 4 & 5 & 6 & 7 & 8 & 9\\ \hline
					\end{tabular}
			\end{table}
			\begin{tikzpicture}[overlay]
				\path[->] (n1) edge [out=0, in=90] (t1);
				\path[->] (n2) edge [out=0, in=90] (t2);
				\path[->] (n3) edge [out=-90, in=-90] node[below] {\footnotesize{menor}} (t3);
			\end{tikzpicture}
		\end{frame}

		\begin{frame}
		\frametitle{\secname}
		\framesubtitle{\subsecname}
				
			\begin{description}
				\item[gap] = 1; 
				\item[swap] = false; 
				\item[i] = 3; \tikz\node[coordinate] (n2) {};
				\item[j] = i + gap = 4; \tikz\node[coordinate] (n1) {};
			\end{description}

			\begin{table}[!htb]
				\centering
					\begin{tabular}{|l|c|c|c|c|c|c|c|c|c|c|}
						\hline
						\rowcolor{chameleongreen3}Posição     & 0 & 1 & 2 & \tikz\node(t2){3}; & \tikz\node(t1){4}; & 5 & 6 & 7 & 8 & 9\\ \hline
						\rowcolor{chameleongreen2}\cellcolor{chameleongreen3}Valor       & 0 & 1 & 2 & \tikz\node(n3){3}; & \tikz\node(t3){4}; & 5 & 6 & 7 & 8 & 9\\ \hline
					\end{tabular}
			\end{table}
			\begin{tikzpicture}[overlay]
				\path[->] (n1) edge [out=0, in=90] (t1);
				\path[->] (n2) edge [out=0, in=90] (t2);
				\path[->] (n3) edge [out=-90, in=-90] node[below] {\footnotesize{menor}} (t3);
			\end{tikzpicture}
		\end{frame}

		\begin{frame}
		\frametitle{\secname}
		\framesubtitle{\subsecname}
				
			\begin{description}
				\item[gap] = 1; 
				\item[swap] = false; 
				\item[i] = 4; \tikz\node[coordinate] (n2) {};
				\item[j] = i + gap = 5; \tikz\node[coordinate] (n1) {};
			\end{description}

			\begin{table}[!htb]
				\centering
					\begin{tabular}{|l|c|c|c|c|c|c|c|c|c|c|}
						\hline
						\rowcolor{chameleongreen3}Posição     & 0 & 1 & 2 & 3 & \tikz\node(t2){4}; & \tikz\node(t1){5}; & 6 & 7 & 8 & 9\\ \hline
						\rowcolor{chameleongreen2}\cellcolor{chameleongreen3}Valor       & 0 & 1 & 2 & 3 & \tikz\node(n3){4}; & \tikz\node(t3){5}; & 6 & 7 & 8 & 9\\ \hline
					\end{tabular}
			\end{table}
			\begin{tikzpicture}[overlay]
				\path[->] (n1) edge [out=0, in=90] (t1);
				\path[->] (n2) edge [out=0, in=90] (t2);
				\path[->] (n3) edge [out=-90, in=-90] node[below] {\footnotesize{menor}} (t3);
			\end{tikzpicture}
		\end{frame}

		\begin{frame}
		\frametitle{\secname}
		\framesubtitle{\subsecname}
				
			\begin{description}
				\item[gap] = 1; 
				\item[swap] = false; 
				\item[i] = 5; \tikz\node[coordinate] (n2) {};
				\item[j] = i + gap = 6; \tikz\node[coordinate] (n1) {};
			\end{description}

			\begin{table}[!htb]
				\centering
					\begin{tabular}{|l|c|c|c|c|c|c|c|c|c|c|}
						\hline
						\rowcolor{chameleongreen3}Posição     & 0 & 1 & 2 & 3 & 4 & \tikz\node(t2){5}; & \tikz\node(t1){6}; & 7 & 8 & 9\\ \hline
						\rowcolor{chameleongreen2}\cellcolor{chameleongreen3}Valor       & 0 & 1 & 2 & 3 & 4 & \tikz\node(n3){5}; & \tikz\node(t3){6}; & 7 & 8 & 9\\ \hline
					\end{tabular}
			\end{table}
			\begin{tikzpicture}[overlay]
				\path[->] (n1) edge [out=0, in=90] (t1);
				\path[->] (n2) edge [out=0, in=90] (t2);
				\path[->] (n3) edge [out=-90, in=-90] node[below] {\footnotesize{menor}} (t3);
			\end{tikzpicture}
		\end{frame}

		\begin{frame}
		\frametitle{\secname}
		\framesubtitle{\subsecname}
				
			\begin{description}
				\item[gap] = 1;
				\item[swap] = false;  
				\item[i] = 6; \tikz\node[coordinate] (n2) {};
				\item[j] = i + gap = 7; \tikz\node[coordinate] (n1) {};
			\end{description}

			\begin{table}[!htb]
				\centering
					\begin{tabular}{|l|c|c|c|c|c|c|c|c|c|c|}
						\hline
						\rowcolor{chameleongreen3}Posição     & 0 & 1 & 2 & 3 & 4 & 5 & \tikz\node(t2){6}; & \tikz\node(t1){7}; & 8 & 9\\ \hline
						\rowcolor{chameleongreen2}\cellcolor{chameleongreen3}Valor       & 0 & 1 & 2 & 3 & 4 & 5 & \tikz\node(n3){6}; & \tikz\node(t3){7}; & 8 & 9\\ \hline
					\end{tabular}
			\end{table}
			\begin{tikzpicture}[overlay]
				\path[->] (n1) edge [out=0, in=90] (t1);
				\path[->] (n2) edge [out=0, in=90] (t2);
				\path[->] (n3) edge [out=-90, in=-90] node[below] {\footnotesize{menor}} (t3);
			\end{tikzpicture}
		\end{frame}

		\begin{frame}
		\frametitle{\secname}
		\framesubtitle{\subsecname}
				
			\begin{description}
				\item[gap] = 1; 
				\item[swap] = false; 
				\item[i] = 7; \tikz\node[coordinate] (n2) {};
				\item[j] = i + gap = 8; \tikz\node[coordinate] (n1) {};
			\end{description}

			\begin{table}[!htb]
				\centering
					\begin{tabular}{|l|c|c|c|c|c|c|c|c|c|c|}
						\hline
						\rowcolor{chameleongreen3}Posição     & 0 & 1 & 2 & 3 & 4 & 5 & 6 & \tikz\node(t2){7}; & \tikz\node(t1){8}; & 9\\ \hline
						\rowcolor{chameleongreen2}\cellcolor{chameleongreen3}Valor       & 0 & 1 & 2 & 3 & 4 & 5 & 6 & \tikz\node(n3){7}; & \tikz\node(t3){8}; & 9\\ \hline
					\end{tabular}
			\end{table}
			\begin{tikzpicture}[overlay]
				\path[->] (n1) edge [out=0, in=90] (t1);
				\path[->] (n2) edge [out=0, in=90] (t2);
				\path[->] (n3) edge [out=-90, in=-90] node[below] {\footnotesize{menor}} (t3);
			\end{tikzpicture}
		\end{frame}

		\begin{frame}
		\frametitle{\secname}
		\framesubtitle{\subsecname}
				
			\begin{description}
				\item[gap] = 1; 
				\item[swap] = false; 
				\item[i] = 8; \tikz\node[coordinate] (n2) {};
				\item[j] = i + gap = 9; \tikz\node[coordinate] (n1) {};
			\end{description}

			\begin{table}[!htb]
				\centering
					\begin{tabular}{|l|c|c|c|c|c|c|c|c|c|c|}
						\hline
						\rowcolor{chameleongreen3}Posição     & 0 & 1 & 2 & 3 & 4 & 5 & 6 & 7 & \tikz\node(t2){8}; & \tikz\node(t1){9};\\ \hline
						\rowcolor{chameleongreen2}\cellcolor{chameleongreen3}Valor       & 0 & 1 & 2 & 3 & 4 & 5 & 6 & 7 & \tikz\node(n3){8}; & \tikz\node(t3){9};\\ \hline
					\end{tabular}
			\end{table}
			\begin{tikzpicture}[overlay]
				\path[->] (n1) edge [out=0, in=90] (t1);
				\path[->] (n2) edge [out=0, in=90] (t2);
				\path[->] (n3) edge [out=-90, in=-90] node[below] {\footnotesize{menor}} (t3);
			\end{tikzpicture}
		\end{frame}